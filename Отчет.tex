\documentclass[12pt]{article}
\usepackage[a4paper, left=5mm, right=5mm, top=5mm, bottom=5mm]{geometry}
%\usepackage[a4paper, top=15mm, right=10mm, bottom=10mm, left=10mm]{geometry}
\usepackage[russian]{babel}
\usepackage{fontspec}
\usepackage{graphicx}
\usepackage[unicode]{hyperref}
\usepackage{enumitem}
\usepackage{wrapfig}
\usepackage{tabularx}
\usepackage{amssymb}
\usepackage{gensymb}
\usepackage{amsmath}
\usepackage{blindtext}
\usepackage{float}
\usepackage{multicol}
\usepackage{latexsym}
%\usepackage[font={bf}, name={Рис. }, justification=justified]{caption}
\usepackage{caption}
\usepackage{subcaption}
\usepackage{listings}
\usepackage{xcolor}
\usepackage{breqn}
\usepackage{pdfpages}
\usepackage{verbatim}

\setmainfont{Times New Roman}
%\righthyphenmin=2 % правильные переносы
%\graphicspath{{images/}} % путь к картинкам

\title{Lab6 OPD}
\author{Азат Сиразетдинов}

\begin{document}
	\thispagestyle{empty}
	\begin{center}
		Федеральное государственное автономное образовательное учреждение\\ 
		высшего образования\\
		«Национальный исследовательский университет ИТМО»\\
		\textit{Факультет Программной Инженерии и Компьютерной Техники}\\
	\end{center}
	\vspace{2cm}
	\begin{center}
		\large
		\textbf{Лабораторная работа № 4}\\
		по дисциплине базы данных\\
		Индексы\\
		Вариант № 1678
	\end{center}
	\vspace{7cm}
	\begin{flushright}
		Выполнил:\\
		cтудент  группы P3116\\
		Сиразетдинов А. Н\\
		Преподаватель: \\
		Горбунов М. В.\\
	\end{flushright}
	\vspace{6cm}
	\begin{center}
		г. Санкт-Петербург\\
		2023г.
	\end{center}
	\newpage
	
	\tableofcontents
	\newpage
	
	\section{Текст задания}
	Составить запросы на языке SQL (пункты 1-2).
	
	Для каждого запроса предложить индексы, добавление которых уменьшит время выполнения запроса (указать таблицы/атрибуты, для которых нужно добавить индексы, написать тип индекса; объяснить, почему добавление индекса будет полезным для данного запроса).
	
	Для запросов 1-2 необходимо составить возможные планы выполнения запросов. Планы составляются на основании предположения, что в таблицах отсутствуют индексы. Из составленных планов необходимо выбрать оптимальный и объяснить свой выбор.
	Изменятся ли планы при добавлении индекса и как?
	
	Для запросов 1-2 необходимо добавить в отчет вывод команды EXPLAIN ANALYZE [запрос]
	
	Подробные ответы на все вышеперечисленные вопросы должны присутствовать в отчете (планы выполнения запросов должны быть нарисованы, ответы на вопросы - представлены в текстовом виде).
	
	Сделать запрос для получения атрибутов из указанных таблиц, применив фильтры по указанным условиям:
	Н_ТИПЫ_ВЕДОМОСТЕЙ, Н_ВЕДОМОСТИ.
	Вывести атрибуты: Н_ТИПЫ_ВЕДОМОСТЕЙ.ИД, Н_ВЕДОМОСТИ.ЧЛВК_ИД.
	Фильтры (AND):
	\begin{enumerate}[]\item
	\item Н_ТИПЫ_ВЕДОМОСТЕЙ.НАИМЕНОВАНИЕ > Ведомость.
	\item Н_ВЕДОМОСТИ.ИД > 1250981.
	\end{enumerate}
	Вид соединения: RIGHT JOIN.
	Сделать запрос для получения атрибутов из указанных таблиц, применив фильтры по указанным условиям:
	Таблицы: Н_ЛЮДИ, Н_ОБУЧЕНИЯ, Н_УЧЕНИКИ.
	Вывести атрибуты: Н_ЛЮДИ.ИМЯ, Н_ОБУЧЕНИЯ.ЧЛВК_ИД, Н_УЧЕНИКИ.НАЧАЛО.
	Фильтры: (AND)
	\begin{enumerate}[]\item 
	\item Н_ЛЮДИ.ФАМИЛИЯ < Ёлкин.
	\item Н_ОБУЧЕНИЯ.ЧЛВК_ИД > 112514.
	\item Н_УЧЕНИКИ.ИД > 100410.
	\end{enumerate}
	Вид соединения: RIGHT JOIN.
	
	\section{Реализация запросов на SQL}
	
	\subsection{Запрос 1}
	\begin{verbatim}
		SELECT
			Н_ТИПЫ_ВЕДОМОСТЕЙ.ИД,
			Н_ВЕДОМОСТИ.ЧЛВК_ИД
		FROM
			"Н_ТИПЫ_ВЕДОМОСТЕЙ"
		RIGHT JOIN "Н_ВЕДОМОСТИ" 
		ON
			Н_ВЕДОМОСТИ."ТВ_ИД" = Н_ТИПЫ_ВЕДОМОСТЕЙ."ИД"
		WHERE
			Н_ТИПЫ_ВЕДОМОСТЕЙ.НАИМЕНОВАНИЕ > 'Ведомость'
			AND Н_ВЕДОМОСТИ.ИД > 1250981;
	\end{verbatim}

	\subsection{Запрос 2}
	\begin{lstlisting}[
		language=SQL,
		showspaces=false,
		basicstyle=\ttfamily,
		numbers=left,
		numberstyle=\tiny,
		commentstyle=\color{gray}
		]
		SELECT
			Н_ЛЮДИ.ИМЯ,
			Н_ОБУЧЕНИЯ.ЧЛВК_ИД,
			Н_УЧЕНИКИ.НАЧАЛО
		FROM
			"Н_ЛЮДИ"
		RIGHT JOIN "Н_ОБУЧЕНИЯ"
		ON
			Н_ЛЮДИ."ИД" = Н_ОБУЧЕНИЯ."ЧЛВК_ИД"
		RIGHT JOIN "Н_УЧЕНИКИ" 
		ON
			Н_УЧЕНИКИ."ЧЛВК_ИД" = Н_ОБУЧЕНИЯ."ЧЛВК_ИД"
			AND Н_УЧЕНИКИ."ВИД_ОБУЧ_ИД" = Н_ОБУЧЕНИЯ."ВИД_ОБУЧ_ИД"
		WHERE 
			Н_ЛЮДИ.ФАМИЛИЯ < 'Ёлкин'
			AND Н_ОБУЧЕНИЯ.ЧЛВК_ИД > 112514
			AND Н_УЧЕНИКИ.ИД > 100410;
	\end{verbatim}
	
\end{document}







